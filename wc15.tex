\documentclass[a4paper,addpoints]{exam}

\usepackage{amsfonts,amsmath,amsthm}
\usepackage[a4paper]{geometry}
\usepackage{graphicx}
\usepackage{hyperref}
\usepackage{float}

\header{CS/MATH 113}{WC15: Graphs}{Spring 2024}
\footer{}{Page \thepage\ of \numpages}{}
\runningheadrule
\runningfootrule

\printanswers

\qformat{{\large\bf \thequestion. \thequestiontitle}\hfill(\totalpoints\ points)}
\boxedpoints

\title{Weekly Challenge 15: Graphs}
\author{CS/MATH 113 Discrete Mathematics}
\date{Spring 2024}

\begin{document}
\maketitle

\begin{questions}
\titledquestion{$n$-doku}
  Let us define an $n$-doku as an $n\times n$ grid which contains all the numbers from $1$ to $n$ inclusive in the following manner.
  \begin{itemize}
  \item Each number appears exactly $n$ times in the grid.
  \item Each number appears exactly once in each row of the grid.
  \item Each number appears exactly once in each column of the grid.
  \end{itemize}
  For example here is a 4-doku.
  
  \begin{center}
    \begin{tabular}{|c|c|c|c|} \hline 
      4&  3&  1& 2\\ \hline
      3&  4&  2& 1\\ \hline 
      2&  1&  4& 3\\ \hline 
      1&  2&  3& 4\\ \hline 
    \end{tabular}
  \end{center}
  
  \begin{parts}
  \part[2] Below is a partially completed 5-doku.
    
    \begin{center}
      \begin{tabular}{|c|c|c|c|c|} \hline 
        1&  2&  5&  3& 4\\ \hline 
        3&  5&  2&  4& 1\\ \hline 
        5&  1&  4&  2& 3\\ \hline 
         &  &  &  & \\ \hline 
         &  &  &  & \\ \hline
      \end{tabular}
    \end{center}
    
    Copy and complete the 5-doku.
    \begin{solution}
      % Enter your solution here.
    \end{solution}
    
  \part[4] Show that filling in the next row of an $n$-doku is equivalent to finding a matching in some 2n-vertex bipartite graph.
    \begin{solution}
      % Enter your solution here.
    \end{solution}
    
  \part[4] Prove that a matching must exist in this bipartite graph and, consequently, that an incomplete $n$-doku can always be completed.
    \begin{solution}
      % Enter your solution here.
    \end{solution}
  \end{parts}
\end{questions}

\end{document}
%%% Local Variables:
%%% mode: latex
%%% TeX-master: t
%%% End: