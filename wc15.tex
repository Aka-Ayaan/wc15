\documentclass[a4paper,addpoints]{exam}

\usepackage{amsfonts,amsmath,amsthm}
\usepackage[a4paper]{geometry}

\header{CS/MATH 113}{WC15: Matching in Bipartite Graphs}{Spring 2024}
\footer{}{Page \thepage\ of \numpages}{}
\runningheadrule
\runningfootrule

\printanswers

\qformat{{\large\bf \thequestion. \thequestiontitle}\hfill(\totalpoints\ points)}
\boxedpoints

\title{Weekly Challenge 15: Matching in Bipartite Graphs}
\author{CS/MATH 113 Discrete Mathematics}
\date{Spring 2024}

\begin{document}
\maketitle

\begin{questions}
\titledquestion{$n$-doku}
  Let us define an $n$-doku as an $n\times n$ grid which contains all the numbers from $1$ to $n$ inclusive in the following manner.
  \begin{itemize}
  \item Each number appears exactly $n$ times in the grid.
  \item Each number appears exactly once in each row of the grid.
  \item Each number appears exactly once in each column of the grid.
  \end{itemize}
  For example here is a 4-doku.
  
  \begin{center}
    \begin{tabular}{|c|c|c|c|} \hline 
      4&  3&  1& 2\\ \hline
      3&  4&  2& 1\\ \hline 
      2&  1&  4& 3\\ \hline 
      1&  2&  3& 4\\ \hline 
    \end{tabular}
  \end{center}
  
  \begin{parts}
  \part[2] Below is a partially completed 5-doku.
    
    \begin{center}
      \begin{tabular}{|c|c|c|c|c|} \hline 
        1&  2&  5&  3& 4\\ \hline 
        3&  5&  2&  4& 1\\ \hline 
        5&  1&  4&  2& 3\\ \hline 
         &  &  &  & \\ \hline 
         &  &  &  & \\ \hline
      \end{tabular}
    \end{center}
    
    Copy and complete the 5-doku.
    \begin{solution}
      % Enter your solution here.
      \begin{center}  
        \begin{tabular}{|c|c|c|c|c|} \hline 
          1&  2&  5&  3& 4\\ \hline 
          3&  5&  2&  4& 1\\ \hline 
          5&  1&  4&  2& 3\\ \hline 
          4&  3&  1&  5& 2\\ \hline 
          2&  4&  3&  1& 5\\ \hline
        \end{tabular}
      \end{center}  
    \end{solution}
    
  \part[4] Show that filling in the next row of an $n$-doku is equivalent to finding a matching in some 2n-vertex bipartite graph.
    \begin{solution}
      % Enter your solution here.
      \begin{proof}
        Consider a bipartite graph $G = (V,E)$ where there are two disjoint sets of vertices in $V$, $A = \{a_1, a_2, \ldots , a_n\}$ and $B = \{b_1, b_2, \ldots , a_n\}$ such that $|A|=|B|$. \\
        For each $i \in \{1, 2, \ldots , n\}$, there is an edge between $a_i$ and $b_j$ if and only if the $i$th column of the $n$-doku is missing the number $j$. \\
        Then, finding a matching in this bipartite graph is equivalent to filling in the next row of the $n$-doku. \\
        Therefore, filling in the next row of an $n$-doku is equivalent to finding a matching in some 2n-vertex bipartite graph.
      \end{proof}
    \end{solution}
    
  \part[4] Prove that a matching must exist in this bipartite graph and, consequently, that an incomplete $n$-doku can always be completed.
    \begin{solution}
      % Enter your solution here.
      \begin{proof}
        Applying Hall's Marriage Theorem to the bipartite graph defined in part(b): \\
        Let $S \subseteq A$. We need to show that $|N(S)| \geq |S|$ for every subset $S$ of $A$. \\
        Consider any subset $S$ of $A$. Each mumber in $S$ needs to be placed in a different position in $B$ to satisfy the rules. \\
        Since teach position in $B$ can only be occupied by one number, the neighbourhood $N(S)$ contains atleast $|S|$ distict positions i.e. $|N(S)| \geq |S|$. \\
        This satisfies the theorem implying that a perfect matching is guaranteed. This perfect matching corresponds to a valid completion of the $n$-doku. \\
        Therefore, an incomplete $n$-doku can always be completed in this manner.
      \end{proof}
    \end{solution}
  \end{parts}
\end{questions}

\end{document}
%%% Local Variables:
%%% mode: latex
%%% TeX-master: t
%%% End: